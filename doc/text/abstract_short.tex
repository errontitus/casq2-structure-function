Mutations in the calcium-binding protein calsequestrin cause a highly lethal familial arrhythmia, catecholaminergic polymorphic ventricular tachycardia (CPVT). In vivo, calsequestrin multimerizes into filaments, but a compelling atomic-resolution structure of a calsequestrin filament is lacking. We report a crystal structure of a cardiac calsequestrin filament with supporting mutation analysis provided by an \textit{in vitro} filamentation assay. We also report and characterize a novel disease-associated calsequestrin mutation, S173I, which localizes to the filament-forming interface. In addition, we show that a previously reported dominant disease mutation, K180R, maps to the same multimerization surface. Both mutations disrupt filamentation, suggesting that dominant disease arises from defects in multimer formation. A ytterbium-derivatized structure pinpoints multiple credible calcium sites at filament-forming interfaces, explaining the atomic basis of calsequestrin filamentation in the presence of calcium. This work advances our understanding of calsequestrin biochemistry and provides a unifying structure-function molecular mechanism by which dominant-acting calsequestrin mutations provoke lethal arrhythmias.