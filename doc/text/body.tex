% For sections, use \section instead of \section* if you want a TOC entry.

%%%%%%%%%%%%%%%%%%%%%%%%%%%%%%%%%%%%%%%%%%%%%%%%%%%%%%%%%%%%
%%% INTRODUCTION
%%%%%%%%%%%%%%%%%%%%%%%%%%%%%%%%%%%%%%%%%%%%%%%%%%%%%%%%%%%%
%\section{Introduction} 
%
\noindent The \ch{Ca^2+} ion is a ubiquitous chemical messenger in eukaryotic cells, where transient changes in intracellular calcium trigger a diverse set of signal transduction pathways. In accordance with this signaling role, the calcium gradient across the plasma membrane dwarfs the gradients of the other common ions, ranging over 5 orders of magnitude from approximately \SI{2}{\milli\Molar} extracellularly to \SI{100}{\nano\Molar} intracellularly \cite{Carafoli2016-hx}. In addition, many critical signaling pathways involve calcium efflux from the endoplasmic reticulum (ER), an intracellular organelle that maintains an ionic milieu resembling extracellular conditions. Direct physical coupling between the ER lumen and the extracellular space allows some degree of passive equilibration between these two compartments \cite{Prakriya2015-vs}, so that the biochemical work of establishing and maintaining calcium stores is minimized.

In muscle, the transduction of an electrical signal at the membrane into a calcium flow that activates the contractile apparatus is known as excitation-contraction coupling. Under conditions of high muscle loading, calcium release from the sarcoplasmic reticulum (SR, a muscle-specific extension of the ER) becomes much more substantial than other biological calcium fluxes. In order to minimize the energy expended per contractile cycle, muscle cells possess enhanced calcium buffering at the sites of storage and release, allowing a high total calcium content with much lower free calcium. Calsequestrin (CSQ) is a 44 kD highly acidic protein that serves as the principal SR calcium buffer of cardiac and skeletal muscle, storing up to \SI{50}{\percent} of SR calcium in a bound state, with each calsequestrin monomer storing up to 40-50 calcium ions \cite{MacLennan1971-fw}; \cite{MacLennan1974-rv}; \cite{Ostwald1974-yr}; \cite{Costello1986-ma}; \cite{Franzini-Armstrong1987-xb}; \cite{Wang1998-cm}; \cite{Park2004-bu}; \cite{Knollmann2006-vy}. Calsequestrin is complexed to the SR calcium channel, the ryanodine receptor (RyR), thereby ensuring that calcium is stored near the site of release \cite{Bers2004-ns}. % Michael asks about the timescale - "does this help over long times? or is your idea that you might relax after a while and SERCA can catch up? I don't follow." I think CASQ2 is fast-off, consistent with its job of localizing Ca. I *also* think that CASQ2 is *effectively* a slow calcium buffer in that it slowly polymerizes and localizes calcium (consistent with rest potentiation). 

Since their initial identification, in 1971, a substantial field of research has formed around calsequestrins. Calsequestrins (CASQ1 in skeletal muscle and CASQ2 in cardiac muscle) are highly homologous in structure and function, and \SI{64}{\percent} identical in sequence, with skeletal muscle calsequestrin appearing to have higher calcium capacity \cite{Park2004-bu}. Calcium-binding propensity is explained in large part by calsequestrin's remarkable fraction of negatively-charged acidic residues (\SI{26}{\percent} glutamate or aspartate in CASQ2 and \SI{28}{\percent} in CASQ1, with corresponding average isoelectric points of 4.2 and 4.0, respectively). In both cardiac and skeletal muscle, calsequestrin localizes to the junctional SR (jSR) of muscle and forms multimers that are anchored to the luminal SR membrane. 
\begin{hlbreakable} 
Low resolution electron micrographs of isolated skeletal muscle fibers reveal the possibility of more than one multimer type: in some studies a filamentous morphotype is clearly discernible, whereas in others a dense, reticular condensate is observed \cite{Franzini-Armstrong1987-xb}; \cite{Perni2013-ew}.
\end{hlbreakable}
Multimers are anchored to a complex consisting of RyR and the single-pass transmembrane proteins triadin and junctin \cite{Bers2004-ns}. Extended homo-multimerization provides high-density calcium storage, but multimerization appears to be essential for localization, too, in that multimerization-defective mutants are trafficked along the secretory pathway and lost from the ER/SR \cite{Milstein2009-ig}; \cite{McFarland2010-yi}; \cite{Knollmann2010-fl}. Additional insights into calsequestrin biochemistry and cell biology encompass post-translational modifications \cite{Sanchez2012-yl}; \cite{Kirchhefer2010-ui}, calcium-storage capacity \cite{Park2004-bu}, and interactions with the calcium release unit \cite{Zhang1997-dy}; \cite{Rani2016-ql};\cite{Handhle2016-mz}.

Calsequestrin's relevance to human disease is well established. Mutations in skeletal muscle calsequestrin have been putatively linked to malignant hyperthermia and to vacuolar myopathies \cite{Lewis2015-zv}, while mutations in cardiac calsequestrin are well known to cause catecholaminergic polymorphic ventricular tachycardia (CPVT), a highly lethal familial arrhythmia. CPVT is caused by a state of cardiac calsequestrin deficiency, whether arising from null/hypomorphic alleles or from point mutants that disrupt calsequestrin multimerization and localization \cite{Bal2010-vf}; \cite{Bal2011-tv}. 
\begin{hlbreakable}
Most CPVT-causing \textit{CASQ2} point mutations have recessive inheritance, which is concordant with a mechanism whereby deficiency leads to disease. For a protein whose function depends on multimerization, there is, at least to date, a surprising paucity of mutations with putative dominant negative mechanism. The sole known \textit{CASQ2} disease mutation with strong evidence for dominant inheritance, K180R, was not reported to cause defects in multimerization and was proposed to act via as-yet undetermined alternative mechanisms \cite{Gray2016-kx}. The biochemical mechanism for dominant-acting \textit{CASQ2} disease mutations thus remains an open question.
\end{hlbreakable}

Despite the rich body of work concerning calsequestrin biology, no compelling high-resolution candidate structure for a calsequestrin filament has emerged. Although sixteen crystal structures of calsequestrins have been published \cite{Wang1998-cm}; \cite{Park2004-bu}; \cite{Kim2007-sj}; \cite{Sanchez2012-yl}; \cite{Sanchez2012-qi}; \cite{Lewis2015-zv}; \cite{Lewis2016-am}, none of these contain a convincing filament-like assembly. All 16 prior structures reveal calsequestrin dimers that are nearly-identical to one another, but a search for dimer-to-dimer interfaces within and across these crystal unit cells reveals only weak crystallographic packing contacts that appear incompatible with robust biological multimerization. In addition, the observed dimer-to-dimer interfaces vary substantially from one lattice to another. Critically, a lack of mutagenesis studies supporting proposed oligomerization interfaces in the prior published structures calls into question whether the relevant biological multimer has ever been observed. Cumulatively, the prior studies have established that calsequestrins dimerize in a wide variety of conditions, with \textit{intra}-dimer interactions that are largely the same across published structures. The mechanism by which dimers assemble into higher order multimers (\textit{inter}-dimer assembly) remains elusive.

A structure of a calsequestrin filament (or other relevant high-molecular weight multimer) would advance our understanding of calsequestrin biology and permit a comprehensive mapping of disease-causing mutations to biologically-relevant multimerization interfaces. We report here an investigation of dominant-acting cardiac calsequestrin disease mutants that has culminated in a new X-ray structure of cardiac calsequestrin, one that we believe reveals a biologically relevant filament-forming interface. We show that known dominant-acting mutants - one previously reported, and one described for the first time in this study - map to the newly reported multimerization interface. Furthermore, we provide supporting biochemical analysis of likely calcium-binding sites where the protein multimerizes. 
\begin{hlbreakable}
Finally, we provide a plausible mechanism by which point mutations at different filament-forming interfaces may lead to different disease inheritance patterns, revealing how a group of mutations with shared biochemical features (multimerization defects) may have markedly different cell biological effects. These findings fundamentally advance our understanding of the mechanism by which calsequestrin contributes to calcium homeostasis and provide a missing link in our understanding of how dominant-acting \textit{CASQ2} mutations cause disease.
\end{hlbreakable}
% Up to this point, we want reader to grasp that 1) filamentation matters for localiztion, therefore filament structure matters, 2) prior filament structures wrong, we have the correct one, and 3) the protein is important for calcium homeostasis and disease. Now we will reveal a family carrying a previously-uncharacterized mutation in cardiac calsequestrin with likely autosomal dominant pathogenicity. Then we will show that this mutation causes a profound defect in cardiac calsequestrin filamentation kinetics which is not well-explained by protein-protein interfaces observed in existing calsequestrin X-ray structures. 


%%%%%%%%%%%%%%%%%%%%%%%%%%%%%%%% CHANGES FOR LATER
% variant, not mutation
% update labels in supplement maps
% color in pedigree? collapse uninformative regions? replace uninformative family units in the pedigree with diamonds.
% order of panels in fig 6.

% axis rotator should be more modular?
% this is what I need to do tikz labeling using offsets relative to the size of the figure: https://gist.github.com/Caramdir/899107
% clean up. Copy everything to a new private repository not called DRAFT.
% include MTZ map coeffs file in github?


%%%%%%%%%%%%%%%%%%%%%%%%%%%%%%%% TODO

% DONE: Final # of Yb atoms.
% DONE: Final BSA in dimer figure.
% PDB deposition:
% Decide final BIOMT stuff. What to do with BIOMT in anomalous structure? Leave out?
% submit cif instead of PDB.
% fix data processing -was dials, not xds
% try the mmtbx_deposition tool to add in the sequence file?
% update biomt for both native and anom. I think native is correct but need to choose a sym operature for anom.
% update publication record in both PDB and in Zenodo.
% provide CC-half in highest resolution shell.
% provide related experimental dataset PDB code AND Zenodo DOIs.
% update related entries
% update DOIs
% update publication record in both PDB and in Zenodo.

%%%%%%%%% Review lit for implications
% https://www.ncbi.nlm.nih.gov/pubmed/31646517
% https://www.ncbi.nlm.nih.gov/pubmed/31646516
% https://www.ncbi.nlm.nih.gov/pubmed/31646515
% https://www.ncbi.nlm.nih.gov/pubmed/31646511
% Plus there was one other recent Ca release modeling paper.


%%%%%%%%%%%%%%%%%%%%%%%%%%%%%%%% After publication
% update PDB with DOI, related submission codes, publication info (biorxiv, pubmed)
% update Zenodo - publication record.
% When the primary citation associated with your entry is published, please notify us through the deposition system and provide the PubMed ID (if available), journal name, volume, page numbers, title, authors list and DOI.
% provide publication details to  deposit@wwpdb.org 2 weeks prior.


%%%%%%%%%%%%%%%%%%%%%%%%%%%%%%%% Graduating
% UCSF printing service, 3 copies, Natalia to pay.
% Schedule for printing and seminar are of your own choosing. June/July fine.
% Below is the information that you will need for the binding of your dissertation:
% 3 copies is the usual number of dissertations ordered and paid for by your PI (one copy BMS, one copy PI, one copy for you)
% You’ll need to get a purchase order number/and purchase order through BearBuy from your PI/Lab manager/finance office
% Vendor: Herring and Robinson Binder, 415-468-0440
% http://www.herringandrobinsonbookbinders.com/
% §  Cost per copy – Let Demian know what you choose:
% ·         Dissertations bound all CAPS....$35.00
% ·         Dissertations bound uppercase/lower case.....$40.00
% ·         University seal on front cover.....$6.00
% Get 3 copies, it is .08 per black and white pages and .25 per color pages.
% Choose the color of your binding and lettering on the binding (samples in our office)
% You’re good to go!
% We’ll take care of placing the actual order.  Once the final products have been returned to us, I will contact you on how you’d like me to distribute the various copies.

%%%%%%%%%%%%%%%%%%%%%%%%%%%%%%%%%%%%%%%%%%%%%%%%%%%%%%%%%%%%
%%% RESULTS
%%%%%%%%%%%%%%%%%%%%%%%%%%%%%%%%%%%%%%%%%%%%%%%%%%%%%%%%%%%%
%\section{Results}
% Keeping an outline here makes things easier to organize.
\newcommand{\headingsubsectionone}{Autosomal Dominant \textit{CASQ2} Disease Mutations Disrupt Calsequestrin Multimerization Yet Are Not Well-Explained by Prior Calsequestrin Structures}
\newcommand{\headingsubsectiontwo}{The New Cardiac Calsequestrin Filament Candidate Is Helical at the Domain Level}
\newcommand{\headingsubsectionthree}{The 3-Helix Configuration of the New Filament Candidate Promotes Close-Packing of Thioredoxin Domains}
%
\newcommand{\headingsubsectionfourandfive}{Lanthanide Substitution Reveals the Biochemical Basis of Cation-Driven Filament Assembly}
% We'll use subsubsections for these:
\newcommand{\headingsubsubsectionfour}{Cation Binding Leads to Conformational Shifts in Calsequestrin Dimers}
\newcommand{\headingsubsubsectionfive}{Cations Are Trapped at Inter-Dimer Filament-Forming Interfaces}
%
\newcommand{\headingsubsectionsix}{The Cardiac Calsequestrin Filament Contains a Continuous, Solvent-Accessible Lumen Along Its Long Axis}
\newcommand{\headingsubsectionseven}{Dominant Disease Mutations Disrupt Cardiac Calsequestrin's Filament-Forming Interface}

\subsection{\headingsubsectionone}

We encountered a proband from a family with CPVT-like tachy-arrhythmias and multiple cases of sudden, unexplained death at a young age (Figure~\ref{fig:S173I_genetics_and_biochemistry}a). The proband presented at age 33 with a biventricular tachycardia on ECG. She underwent an electrophysiologic study that revealed an inducible ventricular tachycardia with focal origination next to the anterior fascicle, and an inducible atypical atrioventricular nodal reentry tachycardia that was successfully ablated. Targeted sequencing of channelopathy genes (\textit{KCNQ1}, \textit{KCNH2}, \textit{SCN5A}, \textit{ANK2}, \textit{KCNE1}, \textit{KCNE2}, \textit{KNCJ2}, \textit{CAV3}, \textit{RYR2}, and \textit{CASQ2}) revealed only a heterozygously-carried isoleucine-for-serine substitution at position 173 in cardiac calsequestrin. The proband's son and multiple other family members report a tachycardic phenotype (ranging from self-reported palpitations to diagnosed episodes of tachycardia). The family history is also notable for multiple cases of sudden cardiac death at a young age. Overall, the distribution of affected individuals in the pedigree is potentially consistent with dominant inheritance. As the family did not consent to follow-up genetic testing or clinical phenotyping, we were unable to rigorously assess disease/mutation co-segregation.
% Proband detail: Presented at age 33 with a history of rapid biventricular VT documented on outside ECG.  She was at work at a doctors office, experienced sudden chest discomfort that radiated to arms and back. ECG was done in the office documenting rapid biventricular VT. She was taken to the ER where the arrhythmia spontaneously terminated. She reported another episode of sudden tachycardia with HR of ~130. She underwent EP study that showed Inducible ventricular tachycardia, focal origination next to anterior fascicle and inducible atypical AVNRT that was successfully ablated.
% Son of proband detail: Son presented at age 18 with a reported history of 4-5 brief episodes of rapid palpitations associated with dizziness and occasional numbness but no syncope with no identifiable trigger. Outside ECG, echo, stress test and 15 day event monitor were reportedly normal other than sinus tachycardia. Resting ECG in clinic was normal.  Dr. Scheinman recommended loop recorder but he never followed up.
% Other family: While there is reported history of tachycardia in son and sister but no one else has documented CPVT phenotype. The SCD is either idiopathic or in the setting of CAD/CHD or SIDS and is on both maternal and paternal sides of the family.  And, no co-segregation was able to be done.
% Likely pathogenic per ACMG? But without co-segregation, "novel variant" is a more conservative description.

Since pathogenic point mutations in CASQ2 are known to exhibit defective multimerization, we elected to investigate the S173I mutation biochemically in a simple turbidity assay. 
\begin{hlbreakable}
We chose this approach because turbidometric monitoring of calsequestrin's calcium-induced multimerization is a well-established practice \cite{Bal2010-vf}\cite{Bal2011-tv}. In this assay, the observed increase in turbidity after addition of calcium to purified cardiac calsequestrin is fully reversible with addition of stoichiometric EDTA, indicating that the calcium-induced light-scattering observed in the \textit{in vitro} context is due to patterned assembly rather than non-specific aggregation (Figure~\ref{fig:biochemistry_supplement}a). Although strong genetic evidence for pathogenicity of the S173I variant is lacking, 
\end{hlbreakable}
the turbidity assay for the S173I mutant reveals a profound decrease in multimerization rate (Figure~\ref{fig:S173I_genetics_and_biochemistry}b). Minimal improvement in the multimerization rate is observed in a non-physiologic \SI{0}{\milli\Molar} potassium condition (Figure~\ref{fig:biochemistry_supplement}b), demonstrating that the mutant protein is intact but defective in multimerization.  

The striking effect of S173I on CASQ2 mutimerization, combined with the fact that other known \textit{CASQ2} point mutants exhibit the same biochemical defect, prompted us to reexamine the multimerization capacity of the K180R mutant - to date, the only known \textit{CASQ2} disease mutation with strong evidence for dominant pathogenicity. The turbidity assay for K180R under the same conditions used for S173I shows little difference from the wild type \textit{CASQ2} variant (Figure~\ref{fig:S173I_genetics_and_biochemistry}c). However, prior reports suggested that calsequestrin maintains distinct magnesium and calcium binding sites \cite{Krause1991-le}. Therefore, we investigated multimerization kinetics of the K180R mutant in the presence of magnesium. Strikingly, extended incubation of the K180R mutant with magnesium (2 mM MgCl2) prior to addition of calcium yields a profound multimerization defect (Figure~\ref{fig:S173I_genetics_and_biochemistry}d). In vivo, calsequestrin would likely encounter similar levels of free magnesium throughout the SR. 
% from point mutants that disrupt calsequestrin multimerization.\cite{Bal2010-vf}\cite{Bal2011-tv}

Interestingly, neither S173 nor K180 fall at credible, previously identified candidate multimerization interfaces: they are not near the intra-dimer interface, nor are they near candidate inter-dimer interfaces in the prior crystal structures. As the dominant inheritance pattern is classically associated with disease mutations that disrupt protein-protein interactions, we would have expected to find these and other dominant-acting disease mutations at cardiac calsequestrin's dimer or multimer interfaces. We therefore elected to pursue another structure in the belief that the biologically relevant multimer has not yet been observed. % Michael: Most structure papers don't say it this way. you need to state a different condition, mutant, etc that might get new structures, but don't simple say "we are going to get new structures"

\subsection{\headingsubsectiontwo}

We have determined a new crystal structure of human cardiac calsequestrin obtained from a full-length construct in a very low-pH (3-3.5) crystallization condition. The previously characterized calsequestrin dimer is again observed, but now in an arrangement that produces a closely-packed filament (Figure~\ref{fig:filament_overview}a). Crystallographic data collection and processing statistics are summarized in (Table~\ref{tab:table_xtal_stats}) for the native structure as well as a ytterbium-soaked condition used to identify probable calcium-binding sites. % Avoiding "we report here". Sounds like an abstract. The paper provides the comprehensive report of those initial findings.

The new structure provides a compelling candidate for a biologically-plausible higher-order multimer. The repeating unit of the native crystal resides in a higher-symmetry point group compared to prior calsequestrin structures (Table~\ref{tab:table_xtal_comp}). The oligomer-forming contacts that exist between dimers are novel, differing significantly from all previously reported calsequestrin crystal structures (Figure~\ref{fig:inter_dimer_interface_BSA_comparison}). Furthermore, the candidate filament-forming interfaces in our structure collectively encompass significantly greater buried surface area than observed at any previous calsequestrin inter-dimer interface (Figure~\ref{fig:inter_dimer_interface_BSA_comparison}). 
 
The cardiac calsequestrin monomer, like its skeletal calsequestrin equivalent, consists of an N terminal loop, 3 thioredoxin domains, and a disordered acidic tail. Within the dimer, there is two-fold symmetry with mutual exchange of N-terminal loops, as previously observed (Figure~\ref{fig:filament_overview}b). In our structure, the dimers are stacked along a screw axis to form the filament, with each dimer rotated \ang{90} with respect to its neighbors. Although the dimers are positioned at discrete \ang{90} rotations with respect to one another, the underlying architecture of the filament is in fact helical at the level of thioredoxin domains (Figure~\ref{fig:filament_overview}b). Thioredoxin domains II and III form a double helix at the core of the filament (Figure~\ref{fig:filament_overview}c). An outer helix or "collar" of larger diameter, consisting only of thioredoxin domain I, then winds around the inner double helix. All helices are left-handed, corresponding to the left-handed screw axis at the level of dimer-stacking. 

\subsection{\headingsubsectionthree}
%
The 3-helix configuration appears to promote the close packing of globular thioredoxin domains. Helical packing permits each domain to contact multiple other domains in multiple other protomers, which is in stark contrast to other reported candidate structures (PDB ID 1A8Y, rabbit skeletal muscle calsequestrin; PDB ID 1SJI, canine cardiac calsequestrin), which lack the helical pitch and have much more limited interacting surface area (Figure~\ref{fig:filament_comparison}). The close-packing of the new filament candidate is starkly visible when thioredoxin domains are represented as equally-sized spheres centered at the domain center of mass (Figure~\ref{fig:filament_comparison}, right-hand side). Prior putative calsequestrin filament candidates have fewer inter-domain contacts, fewer inter-protomer contacts, and substantially less buried surface area compared to the new candidate. % Note: since any screw axis symmetry operator will produce a helix, the biophysical significance of this intriguing 3-helix configuration must be subjected to further interrogation.e

Notably, our crystallization condition produced crystals only at a low pH (3-3.5, measured by applying a crystallization drop to litmus paper) resulting from the use of aged PEG reagents (presumably having undergone degradation to glycolic acid) or from direct addition of concentrated HCl. This restriction obtained irrespective of calcium in concentrations ranging from trace to 14 mM (higher calcium concentrations led to precipitation, likely due to the presence of sulfate, and were therefore not assessed). Given calsequestrin’s remarkable overall acidity and low isoelectric point, an explanation for the role of low pH in promoting calsequestrin filamentation could be the neutralization of substantial negative charge on acidic side chains that extend across the filament interfaces. % Lots of interesting and suggestive aspects of the crystal morphology that we won't get into here due to space constraints. In more detail: In an effort to explain why prior calsequestrin crystals did not produce the closely packed filament that we report, we examined crystallization conditions used for prior structures and compared them to our own. Notably, our crystallization condition produced crystals only at low pH (3-3.5) resulting from the use of aged PEG reagents (presumably having undergone degradation to glycolic acid) or from direct addition of concentrated HCl. This restriction obtained irrespective of calcium in concentrations ranging from trace to 14 mM (higher calcium concentrations led to precipitation, likely due to the presence of sulfate, and were therefore not assessed). Crystals that grew in our acidic condition consistently had needle or rod morphology with splinter defects at the ends, suggestive of close packing along the crystal's long axis (likely corresponding to the direction of filament elongation) and weaker interactions in other directions. Given calsequestrin's remarkably overall acidity and low isoelectric point, an obvious explanation for the low pH requirement is neutralization of substantial negative charge on acidic side chains that extend across the oligomer interface.  

\subsection{\headingsubsectionfourandfive}
% 
Prior work has identified putative calcium binding sites at the dimer interface of calsequestrin \cite{Sanchez2012-qi}. Our new structure permits us to re-examine the role of multivalent cations at calsequestrin's intra-dimer surfaces, while also evaluating the presence of cations at the newly-observed inter-dimer surfaces responsible for the higher-order multimer observed in the crystal. To localize candidate calcium ligand sites within the context of the new filament structure, we collected data from a Ytterbium (Yb)-soaked crystal (Table~\ref{tab:table_xtal_stats}). From the anomalous map, we identified approximately 8 sites with strong Yb signal per CASQ2 chain (> 3.8 $\sigma$ in the anomalous map, and positive difference density in the Fo-Fc map). As the prior work to identify calsequestrin's calcium-binding sites was based largely on the indirect method of inference from metal coordination geometry with nearby side chains and waters, the use of Yb provides a direct approach to confirming the prior findings, as well as extending our understanding to the entire filament. % Yb is a multivalent ion with strong anomalous scattering and an ionic radius similar to calcium (cite). The structure we report exhibits multiple calcium ligand sites revealed by ytterbium sustitution. Rare earth metals provide strong anomalous scattering, and their ionic radii make them suitable as isomorphous calcium replacements in protein crystallography \cite{Colman1972-qz}.

\subsubsection{\headingsubsubsectionfour}
% 
We first assessed the presence of Yb at the intra-dimer interface. We identified several high-occupancy Yb sites, most of which are clustered in a narrow region between protomers, where they are coordinated by multiple, highly-conserved acidic residues (Figure~\ref{fig:intra_dimer_interface} with supporting electron density and anomalous signal maps in Figure~\ref{fig:intra_dimer_interface_maps}). Consistent with the location of a putative calcium ion in the prior study, we find a Yb atom coordinated primarily by E147 of chain A and D278 of chain B. In addition, we find a Yb atom coordinated primarily by E143 and E275. Another Yb atom is poorly coordinated by D310 and a putative sulfate anion within a solvent cavity enclosed by the dimer. The low resolution of the derivative structure precludes identification of waters that could contribute to classical coordination geometries, but all identified cation sites are supported by electron density and anomalous signal. Together, these Yb bridging sites stabilize the interaction between thioredoxin domain II from protomer A and thioredoxin domain III of protomer B. The interactions between domains result in close juxtaposition of acidic side chains (Figure~\ref{fig:intra_dimer_interface}a), requiring neutralization of the charge by the multivalent counterions or, in the case of the native structure, protonation. For example, in the native structure, the close interaction of E147/D278 provides an example of a carboxyl-carboxylate bond stabilized by low pH \cite{Sawyer1982-sm}; \cite{Krause1991-le}. Anomalous signal is found at this site in the derivative structure, but consistently on just one side of the otherwise symmetric dimer. This subtle asymmetry, with a cation bound on one side and a carboxyl-carboxylate bond on the other, likely reflects a degree difference in occupancy rather than a sharp distinction, but it is a consistent feature of the density map across all 4 dimers in the crystal asymmetric unit and may be conformationally conducive to filament formation.
% Comparing our Yb sites to the Sanchez 2012, et al paper.
% Focusing on their Fig 4A (typical dimer) and less so on 4C (weird 3UOM asymmetric tetramer). 
% In 4A: 
% E109 in their numbering = E143 in hCasq1 = E128 in hCasq2, not in our ROI.
% D113 in their numbering = D147 in hCasq1 = D132 in hCasq2. Oddly they don't mention this residue. Also not in our ROI.
% E256 in their numbering = E290 in hCasq1 = E275 in hCasq2. This one is consistent with our strongest site (E143/E275)
% In sum, some overlap with our region of interest. 
% In their figure 4C: weird asymmetric tetramer. Ignoring this one.

In both the native structure and Yb-soaked structure, the dimer subunits have undergone substantial inward rotation in comparison to the other high-resolution cardiac calsequestrin structure (PDB ID 1SJI) (Figure~\ref{fig:intra_dimer_interface}b). The inward domain movement is produced largely by rotation of the monomer as a rigid body and results in a conformation where domains II and III are more closely packed. This conformational shift recapitulates a similar finding from the prior study where a \ang{15} rotation within the skeletal calsequestrin dimer was observed for structures crystallized in a high-calcium buffer \cite{Sanchez2012-qi}. Upon inspection of all prior published calsequestrin structures, it is apparent that 6 prior calsequestrin structures belong to this "tightly-packed dimer" group, while the remaining 10 contain dimers that are more loosely-packed, without inward rotation of chains (Figure~\ref{fig:intra_dimer_interface_6OVW_vs_other_overlay}). The more tightly-packed structures were all crystallized in the presence of multivalent cations (usually calcium), or in one case (PDB ID 2VAF) a monovalent cation at extremely high concentration (~\SI{2}{\Molar} \ch{NaCl}). The other group were crystallized with no added multivalent cations, with the exception of PDB ID 3TRP, which contained calcium in the crystallization drop at approximately 5-fold lower than lowest concentration from the tightly-packed group. Within the tightly-packed group, there is a greater degree of conformational disorder in Domain I (Figure~\ref{fig:intra_dimer_interface_6OVW_vs_other_B_factor}). This is accompanied by a modest loss of contact in Domain I, while multiple hydrophobic side chains from Domains II and III that were not buried now obtain buried surface area (Figure~\ref{fig:intra_dimer_interface_msa}). Thus, our data provide additional evidence for conformational change within the dimer upon calcium-binding (likely induced by closer approximation of acidic side chains), independent confirmation of intra-dimer calcium binding sites, and an explanation for why an altered conformation of the dimer becomes energetically tolerable in the presence of calcium. 

% This change in conformation, which we measure as a rigid body rotation of \ang{20} for a single chain, is in fact evident in all calsequestrins crystallized  structure, PDB ID 1SJI, which was crystallized in the absence of multivalent ions (Figure~\ref{fig:intra_dimer_interface}B). The conformational shift recapitulates a similar finding from the prior study where a \ang{15} rotation within the skeletal calsequestrin dimer was observed for structures crystallized in a high-calcium buffer.\cite{Sanchez2012-qi} 

%These 6 structures were all crystallized in the presence of multivalent cations, or in one case a monovalent cations (PDB ID: 2VAF) at extremely high concentration (~\SI{1}{\milli\Molar} \ch{CaCl2}. Nine other calsequestrin structures exhibit no dimer compaction. These 9 were crystallized with no added multivalent cations, with the exception of PDB ID 3TRP, which contained calcium in the crystalliztion drop at approximately 5-fold lower than lowest-concentration condition that produced compaction. 
%The conformational difference results in altered buried surface area at the dimer interface. Within Class I (inwardly-rotated), there is a modest loss of contact in Domain 1, but multiple hydrophobic side chains from Domains 2 and 3 that were not buried in Class II now obtain buried surface area (Figure~\ref{fig:dimerinterface}E, and Figure~\ref{fig:dimerinterfacemsa}).
%The classification of structures into compacted and non-compacted is summarized in (Figure~\ref{fig:dimerinterface}E)
%multivalent cation-containing crystallization conditions in Class I
%, and a rabbit skeletal calsequestrin which contained ~\SI{1}{\milli\Molar} CaCl2 in the crystallization drop
%Compaction of the dimer upon 
%On the basis of this conformational shift, all published calsequestrin structures can be sorted into two conformational classes: class I has inward-rotated subunits, while Class II contains the less-closely packed structures (Table~\ref{tab:tableinterfacedimer}). The conformational difference results in altered buried surface area at the dimer interface. Within Class I (inwardly-rotated), there is a modest loss of contact in Domain 1, but multiple hydrophobic side chains from Domains 2 and 3 that were not buried in Class II now obtain buried surface area (Figure~\ref{fig:dimerinterface}E, and Figure~\ref{fig:dimerinterfacemsa}). Classifying the published structures in this way allows us to extend and confirm the association between neutralization of acidic chains and closer-packing of the dimer. In general, Class I structures were crystallized with high concentrations of multivalent cations, or our case at low pH. In general, Class II structures were crystallized with no added multivalent cations. The only exceptions to this rule are human cardiac calsequestrin crystallized in extremely high monovalent salt concentration, \SI{2}{\Molar} NaCl (PDB: 2VAF), and a rabbit skeletal calsequestrin which contained ~\SI{1}{\milli\Molar} CaCl2 in the crystallization drop (PDB: 3TRP), approximately 5-fold lower than all the multivalent cation-containing crystallization conditions in Class I. Both classes contain a mix of skeletal and cardiac calsquestrin structures across different species, suggesting that the multivalent ion-induced conformational shift is a highly conserved feature of the calsequestrin family as a whole, and does not differ substantially between skeletal and cardiac forms of the protein. 

\subsubsection{\headingsubsubsectionfive}
% 
Like the intra-dimer interface, the inter-dimer filament-forming interface is characterized by closely-apposed acidic side chains. At this interface, acidic surfaces enclose a large solvent cavity with interior and exterior Yb binding sites (Figure~\ref{fig:inter_dimer_interface}a with supporting electron density and anomalous signal maps in Figure~\ref{fig:inter_dimer_interface_maps}). Whereas the negative charge at the dimer interface is accomodated partly by cation coordination and partly by carboxyl-carboxylate bonds, the inter-dimer interface has comparatively higher cation occupancy, measured by magnitude of anomalous signal (Figure~\ref{fig:inter_dimer_interface_maps}). Residues E184 and E187 coordinate Yb atoms with very substantial anomalous peaks near the mouth of the cavity, partially shielded from exposure to the bulk solvent. Within the cavity, residues D144 and E174 appear to weakly coordinate Yb atoms  (the proper geometries of D144 and E174 are difficult to discern, due to lack of side chain density in the electron density maps of both the native and Yb-soaked structures). At the base of cavity, residues D348 and D350 are oriented along with their symmetry mates to form a cluster of 4 acidic side chains that coordinate a single cation. Finally, outside the inter-dimer cavity, on the fully solvent-exposed exterior of the filament, residues D351 and E357 and their symmetry mates form bidentate interactions with two Yb atoms, adopting conformations in which opposing acidic rotamers are bent away from one another, thereby alleviating electrostatic repulsion that would otherwise disrupt multimer formation.

Identification of these Yb coordination sites at the inter-dimer interface provides a testable model for the biological relevance of the new filament structure. To test our model, we mutated the Yb-binding aspartate or glutamate side chains to alanine and examined the effect of these mutations on multimerization kinetics. Mutation of residues D144 and E174 to alanine results in a dramatic increase in multimerization rate (Figure~\ref{fig:inter_dimer_interface}b). Conversely, mutation of residues E184 and E187 to alanine results in a profound multimerization defect (Figure~\ref{fig:inter_dimer_interface}c, left-hand side). Of note, glutamates 184 and 187 belong to an alpha helix that provides a linkage between domains of the outer thioredoxin collar. This helix, belonging to thioredoxin domain II, sits between thioredoxin I domains of different dimers and interacts with inter-dimer salt bridges on either side. Alanine mutagenesis of the D50 residue that participates in a salt bridge with K180 at the N-terminal end of this helix produces a similar defect (Figure~\ref{fig:inter_dimer_interface}c, right-hand side). At the remaining two Yb sites (D348/D350 and D351/E357), mutating acidic residues to alanine would be expected to relieve mutual repulsion of closely-packed acidic side chains. Consistent with this, alanine mutagenesis of these residues has a largely net neutral effect on multimerization kinetics (Figure~\ref{fig:inter_dimer_interface_supplement}a-b). 
% The extended series of ionic interactions between side chains down the length of the helix may cement inter-dimer interactions in two ways. First, this domain 2 helix, with its salt bridges connecting to domain 1s on either side, provides a dimer-to-dimer linkage along the outer thioredoxin collar (Figure~\ref{fig:interdimerinterface}). Second, thioredoxin domain 1 is effectively "glued" to a domain 2 from the other dimer, thus connecting domain 1 to the inner domains but in a cross-linked manner, i.e. connecting to domain 2 on another dimer.
% The most pronounced effects resulted from mutations targeting residues E184 and E187.
% Specifically, residues D144 and E174 appear to weakly coordinate Yb atoms within the cavity (the proper geometries of D144 and E174 are difficult to discern, due to lack of side chain density in the electron density maps of both the native and Yb-soaked structures). 
% residues E184 and E187 coordinate Yb atoms near the mouth of the cavity, partially shielded from exposure to the bulk solvent. Mutation of these residues results in a profound multimerization defect (Figure~\ref{fig:inter_dimer_interface}c, left-hand side). Glutamates 184 and 187 belong to an alpha helix that provides a linkage between domains of the outer thioredoxin collar. This helix, belonging to thioredoxin domain II, sits between thioredoxin I domains of different dimers and interacts with inter-dimer salt bridges on either side. Alanine mutagenesis of the D50 residue that participates in a salt bridge with K180 at the N-terminal end of this helix produces a similar defect (Figure~\ref{fig:inter_dimer_interface}c, right-hand side). % The extended series of ionic interactions between side chains down the length of the helix may cement inter-dimer interactions in two ways. First, this domain 2 helix, with its salt bridges connecting to domain 1s on either side, provides a dimer-to-dimer linkage along the outer thioredoxin collar (Figure~\ref{fig:interdimerinterface}). Second, thioredoxin domain 1 is effectively "glued" to a domain 2 from the other dimer, thus connecting domain 1 to the inner domains but in a cross-linked manner, i.e. connecting to domain 2 on another dimer.

% Two more sites of strong Yb signal are notable. Residues D348 and D350 are oriented along with their symmetry mates to form a cluster of 4 acidic side chains that coordinate a single cation (Figure~\ref{fig:inter_dimer_interface}a and Figure~\ref{fig:inter_dimer_interface_maps}). This Yb site forms the base of the large solvent cavity enclosed by the inter-dimer interface. Mutating the Yb-coordinating residues to alanine would be expected to relieve mutual repulsion of closely-packed acidic side chains. Consistent with this, alanine mutagenesis of these residues has a largely net neutral effect on multimerization kinetics (Figure~\ref{fig:inter_dimer_interface_supplement}a). Outside the inter-dimer cavity, on the fully solvent-exposed exterior of the filament, residues D351 and E357 and their symmetry mates form bidentate interactions with two Yb atoms, adopting conformations in which opposing acidic rotamers are bent away from one another, thereby alleviating electrostatic repulsion that would otherwise disrupt multimer formation (Figure~\ref{fig:inter_dimer_interface}a and Figure~\ref{fig:inter_dimer_interface_maps}). As with the D348/D350 site, the bound cation at this site appears to facilitate filamentation by neutralizing the negatively charged surface of calsequestrin. Alanine mutagenesis of the D351 and E357 residues also has a largely net neutral effect on multimerization kinetics (Figure~\ref{fig:inter_dimer_interface_supplement}b). % Notably, the sites of presumed calcium-binding that we mutated to alanine and for which we identify differences in filamentation kinetics are not involved in candidate oligomerization interfaces of the other major calsquestrin structures. <----- Excluding this sentence unless we bring back the dimer-dimer interface sequence alignment figure to support the claim.

\subsection{\headingsubsectionsix}

% Ideas for a better figure here:
% 1) show a zoom-in of the acidic residues at the pore. Sentence 1: dimer contains a pore, lined by acidic residues. first figure is just the surface showing the pore, then the zoom in.
% 2) A continuous lumn is formed as a result of the fact that each dimer...
Intriguingly, the majority of Yb sites that we identify are located within a continuous solvent cavity that winds through the interior of the filament. The calculated electrostatic potential at the protein surface becomes highly electronegative at the entrance to this cavity, which is lined by acidic residues (Figure~\ref{fig:filament_cavity}a). 
\begin{hlbreakable} 
The small size of this cavity, lined by acidic side chains and then surrounded by the bulk of one of the most acidic proteins known to exist, results in a remarkably electronegative character (Figure~\ref{fig:filament_cavity}b). The acidic side chains of residues D144 and E174 extend far into the high-electronegativity region, possibly explaining why mutating these side chains to alanine resulted in improved multimerization kinetics - in contrast to the effect of alanine mutagenesis of the other interfacial acidic residues, which are much closer to or even outside the boundary of the pocket. \end{hlbreakable} 
A continuous lumen results from the fact that each dimer contains a solvent pocket within its interior, while another solvent pocket is formed at inter-dimer interface. Stacking of dimers thus extends the cavity along the entire length of the filament (Figure~\ref{fig:filament_cavity}c). The calcium ions that would appear to be bound in the dimer's interior could constitute a separate store of calcium that is more slowly-mobilized than the highly-accessible pool of ions bound to the surface and the solvated acidic tail. 

\subsection{\headingsubsectionseven}

The newly observed filament structure permits insight into the disease mechanisms of CPVT-associated calsequestrin variants. Remarkably, both variants shown to exhibit multimerization defect in Figure~\ref{fig:S173I_genetics_and_biochemistry} - namely the arrhythmia-associated S173I variant that initiated this investigation, as well as the afore-mentioned K180R variant recently implicated in CPVT \cite{Gray2016-kx} - affect residues at the inter-dimer interface (Figure~\ref{fig:inter_dimer_interface_cpvt}a). 

Despite the conservative nature of the arginine-for-lysine substitution, strong genetic evidence implicates the K180R variant as pathogenic. We have shown in Figure~\ref{fig:S173I_genetics_and_biochemistry} that the K180R mutation results in a multimerization defect, but only in the presence of physiologic magnesium. Our work provides a mechanistic explanation for this effect. In the filament structure, K180 participates in a salt bridge with D50 at the inter-dimer interface (Figure~\ref{fig:inter_dimer_interface}a, lower-right). However, the fact that magnesium is required to observe the multimerization defect of the K180R mutant suggests that rather than disrupting the salt bridge with D50, K180R alters the electrostatics of the nearby divalent cation-binding site (Figure~\ref{fig:inter_dimer_interface_cpvt}b). The high conservation of K180 in evolution (Figure~\ref{fig:calsequestrin_conservation}) and key role of K180 and its interacting partners in supporting calsequestrin multimerization provide additional support for the physiological relevance of our new structure. 

The dimer-stacking architecture of the filament likewise explains the disruptive effect of the S173I variant. Remarkably, S173 occupies a critical position at the filament-forming interface - a charged pocket formed by the interaction of K87, S173, and D325 (Figure~\ref{fig:inter_dimer_interface_cpvt}c with supporting electron density map in Figure~\ref{fig:inter_dimer_interface_maps_non_liganded}). This pocket, formed of residues from 3 different protomers and also 3 different thioredoxin domains, enforces an interaction between the 3 thioredoxin domains at a single site.  
\begin{hlbreakable}
On the basis of the apparent importance of this site, as well as the absence of D325 in any other previously described candidate filament interface, we mutated residue D325 to D325A. The D325A mutant exhibits similarly depressed multimerization kinetics in the turbidity assay, demonstrating that even a mild hydrophobic substitution in this interfacial pocket has profound effects.
(Figure~\ref{fig:inter_dimer_interface_cpvt}d).
\end{hlbreakable}

%
%%%%%%%%%%%%%%%%%%%%%%%%%%%%%%%%%%%%%%%%%%%%%%%%%%%%%%%%%%%%
%%% DISCUSSION
%%%%%%%%%%%%%%%%%%%%%%%%%%%%%%%%%%%%%%%%%%%%%%%%%%%%%%%%%%%%
%
\section{Discussion}

We report here a crystal structure of a calsequestrin filament. \begin{hlbreakable} Although we do not exclude the possibility that other biologically relevant higher-molecular weight calsequestrin complexes exist, we provide multiple forms of evidence for the biological importance of the filamentous form that we describe. Specifically, we provide biophysical evidence (buried surface area, all-by-all interactions of protomers and domains), biochemical evidence (disruption of multimerization by targeted mutagenesis), and biomedical evidence (disease mutations with dominant inheritance are located at the newly-identified filament-forming interfaces). \end{hlbreakable} We also elucidate the biochemical basis of calsequestrin's divalent cation-induced filamentation. Using Yb substitution, we confirm previous findings of intra-dimer cation binding sites, and we reveal the basis of inter-dimer cation-trapping that appears to govern the rate of multimerization. 
% I don't think we need to go beyond this caveat and describe specific possible alternative structures (e.g. possible branching structures suggested by Franzini-Armstrong. We can let evidence for other structures emerge.)

In addition to replicating the earlier finding that divalent binding promotes tighter packing of the calsequestrin dimer, the crystal structure of the filament provides additional insight into more subtle conformational changes that may favor filamentation. The prevailing view has been that calsequestrin filamentation is driven by an increase in solvent entropy: ions are bound at filament interfaces, and solvent entropy increases as ion hydration shells are lost \cite{Krause1991-le}. Under this view, the net contribution from solvent entropy is much greater than the net enthalpic contribution from calcium binding after considering the energetic cost of dehydration. This hypothesis is complicated, however, by the fact that we observe stable filamentation in the near-absence of ligand (only trace concentrations of divalent cations were present in the low-pH crystallization condition). The low-pH filament is stabilized instead by carboxyl-carboxylate interactions between closely juxtaposed acidic residues. While increased solvent entropy likely contributes to filamentation at some level, we suggest instead that protein conformational entropy is likely to be the major contributing factor. The structure we report is remarkably disordered by conventional B-factor and RSRZ metrics, and especially so in Domain I. In fact, all tightly-packed calsequestrin dimers (7 crystallographic results total, including this study) exhibit increased disorder in the solvent-exposed loops of Domain I (Figure~\ref{fig:intra_dimer_interface_6OVW_vs_other_B_factor}). The consistency of this observation within the group makes it less likely that conformational disorder within Domain I is a feature of our specific crystallization condition and more likely that it is energetic compensation for the dimer's having adopted the more tightly-packed conformation favorable for filamentation. This would be consistent with results from studies of another calcium-binding protein, namely calmodulin, where changes in conformational disorder appear linearly related to the binding energy at its targets \cite{Frederick2007-in}. % This para cites https://www.ncbi.nlm.nih.gov/pubmed/17637663 and could also cite https://www-nature-com.ucsf.idm.oclc.org/articles/nature11271.pdf. Note: we do not use the term free energy here. The filament could be unstable, not an equilibrium entity, and to the extent it is thermodynamically favored, it may be favored only by entropy.

Calsequestrin's pH-sensitivity prompts questions about the possible role of intra-luminal pH changes in the regulation of calcium uptake and release. Carboxyl-carboxylate side chain interactions modulated by prevailing pH are a common feature of calcium-binding proteins, wherein a shared proton provides a stabilizing interaction when the cation is unavailable \cite{Milos1986-kp}; \cite{Krause1991-le}. These carboxyl-carboxylate interactions have a higher pKa than a solitary carboxylate, permitting them to play a significant role at pH ranges closer to physiologic \cite{Sawyer1982-sm}; \cite{Krause1991-le}. Even at neutral pH, studies of calsequestrin and calmodulin have revealed that protons are released upon calcium binding, consistent with loss of carboxyl-carboxylate bonds \cite{Milos1986-kp}; \cite{Krause1991-le}. A small decrease in SR luminal pH during calcium release has been observed \cite{Kamp1998-wc}, and multiple groups have proposed that proton influx into the SR constitutes a small but important fraction of the counterion flow required to maintain charge neutrality when large calcium fluxes occur. Prior work showed a change in calsequestrin's intrinsic fluorescence at a pH of 6.0 \cite{Hidalgo1996-fm}, suggesting that dynamic effects on calsequestrin are not limited to the low pH regime used in our crystallization experiments. Since calsequestrin is present at high concentration and likely acts non-trivially as a proton buffer in its own right, effects due to total free proton concentration may manifest as protons move from an SR-based buffer system to one in the cytosol and back with little change in detectable pH. It is important to note that regulatory effects of dynamic SR pH are speculative and require further elucidation. The calsequestrin monomer/dimer/oligomer equilibrium may be finely tuned across different ER/SR sub-compartments, with calsequestrin appearing to traffic as a mobile monomer until it reaches the jSR \cite{McFarland2010-yi}. A critical concentration or other local conditions may favor oligomer-formation in the jSR. % Notably, the SERCA pump, in addition to being an ATPase, is a \ch{Ca^2+}/\ch{H+} antiporter, so it is reasonable to hypothesize that proton flows over the course of a cycle of CICR have downstream regulatory effects on calsequestrin's filamentation state, or its affinity for calcium, or both. Proton efflux from the SR during calcium reuptake, by way of the SERCA pump's antiporter function, would increase the availability of calcium binding sites. Conversely, protein influx into the SR during calcium release - possibly via a second, independent proton transport pathway within the SERCA pump \cite{Espinoza-Fonseca2017-by} - could conceivably stabilize calsequestrin filaments until high levels of calcium are restored. 
%
% Have chosen to omit the above. It's long, increasingly speculative, and depends on confirmation that at least some significant amount of SERCA isoform is localized to jSR. That may be the case, but it's unclear.

Until recently, \textit{CASQ2} disease mutations were thought to be recessively inherited more or less as a rule, with no strong genetic evidence for a dominant disease allele until the report of the K180R mutation \cite{Gray2016-kx}. The recessive inheritance group corresponds to mutations at or near the intra-dimer interface (e.g. R33Q, D307H, and P308L). Since filamentation is necessary for calsequestrin to remain in the jSR \cite{Milstein2009-ig}; \cite{McFarland2010-yi}; \cite{Knollmann2010-fl}, these intra-dimer interface/filamentation-defective mutants are assumed to be trafficked out of the jSR, leaving any remaining pool of wild type protein largely unaffected. Under this model, filamentation defects overlap mechanistically with a class of calsequestrin-deficient conditions arising from null or hypomorphic alleles, all leading to a final common endpoint of decreased calcium-buffering capacity, and a resulting susceptibility to diastolic calcium leak.  

In contrast to the traditional recessive genetic model of \textit{CASQ2}-associated disease, the mutations investigated in this study (S173I, K180R) are associated with dominant inheritance. These mutations disrupt filamentation at the \textit{inter}-dimer interface instead of the \textit{intra}-dimer interface. The apparent discrepancy in inheritance patterns, mapping cleanly to distinct interfaces albeit with limited sampling, is striking and demands explanation. 
\begin{hlbreakable}
It is tempting to propose that multimerization is poisoned in classical dominant-negative fashion, whereby dimers that contain a mutant protomer remain in the jSR and interfere with assembly. However, this disregards what we have learned from studies of calsquestrin trafficking out of the ER/SR: unincorporated dimers should be just as susceptible as monomers to jSR export. Instead, we propose a combined mechanism that is more complex than a classical dominant negative effect: for as long as mutant-containing dimers may be present in the jSR, they interfere with multimerization, but crucially, when they leave, they steal half of the total WT protein with them. The result is paradoxical and perhaps somewhat novel: insufficiency by way of underlying dominant negative biochemistry (Figure~\ref{fig:graphical_summary}).
\end{hlbreakable}

\begin{hlbreakable}
We favor a defect in protein localization as the common underlying etiology over the alternative explanation that the reduced buffering power of dissociated calsequestrin dimers is sufficient on its own to cause penetrant disease. Only 7 Yb atoms (28 total per tetrameric assembly) are localized to the inter-dimer interface. If this interfacial/total cation ratio is similar for calcium, then the buffering power of dissociated dimers should remain approximately \SI{75}{\percent} of the filament buffering power. Although a number of factors could affect the in vivo buffering capacity of the filament, the fact that the unobserved acidic tail of each calsequestrin monomer is thought to bind close to half of the total bound cation pool \cite{Park2004-bu} means that in the most likely scenario, the true interfacial fraction is even lower. 
\end{hlbreakable}
In sum, taking into account calsequestrin's demonstrated susceptibility to trafficking, we can combine the present work with the existing rich body of calsequestrin research to explain the inheritance puzzle associated with CPVT-causing calsequestrin mutations. Dimer-defective mutants produce monomers which are trafficked away, but the heterozygous state is rescued because WT protein is unaffected by the pool of defective monomeric protein. In contrast, mutations that interfere with the inter-dimer interaction result in depletion of a substantial fraction of WT protein, as unincorporated dimers containing a mix of WT and mutant protein are continually lost. Future work should seek to confirm this hypothetical mechanism using cell biological investigative methods.

