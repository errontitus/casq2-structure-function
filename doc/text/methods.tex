%\clearpage
%
%%%%%%%%%%%%%%%%%%%%%%%%%%%%%%%%%%%%%%%%%%%%%%%%%%%%%%%%%%%%
%%% METHODS
%%%%%%%%%%%%%%%%%%%%%%%%%%%%%%%%%%%%%%%%%%%%%%%%%%%%%%%%%%%%
%
\section{Materials and Methods}

\subsection*{Human Subjects}
The patient included in the study provided informed consent as part of a research protocol approved by the University of California, San Francisco Committee on Human Research. All procedures performed in studies involving human participants were in accordance with the ethical standards of the institutional research committee and with the 1964 Helsinki declaration and its later amendments or comparable ethical standards.

\subsection*{Cloning and Generation of Plasmids}
Full-length cardiac calsequestrin was cloned from human cardiac mRNA by reverse transcription, PCR, A-tailing of a PCR product, and TA ligation. A clone lacking the signal peptide sequence was subcloned by PCR and Gibson Assembly into a T7-based bacterial overexpression vector (pET28a) in front of a 6His site and TEV protease cleavage sequence. Point mutants were generated using the protocol from the Q5 Site-Directed Mutagenesis Kit (New England BioLabs), using either the Q5 or Phusion polymerases. All constructs were transformed in NEB Stable or XL-1 Blue \textit{E. coli}, purified by miniprep, verified by Sanger sequencing, and retransformed into Rosetta (DE3)pLysS \textit{E. coli} for overexpression. Primers used for cloning and mutagenesis are provided in (\supplementarytable~\ref{tab:tableoligos}). Selection for the pET28a vector was performed using \SI{50}{\micro\gram/\liter} kanamycin. Selection for pLysS was performed by adding \SI{25}{\micro\gram/\liter} chloramphenicol. Plasmids are available upon request to the corresponding author.

\subsection*{Expression and Purification of Cardiac Calsequestrin}
pET28a-based expression constructs were transformed into Rosetta (DE3)pLysS \textit{E. coli}. Overnight starter cultures were used to inoculate large cultures (typically \SI{750}{\milli\liter} of broth per \SI{2.8}{\liter} flask), which were grown to OD 0.4 and then induced with \SI{0.25}{\milli\Molar} IPTG. Upon induction, temperature was reduced from \SI{37}{\degreeCelsius} to \SI{24}{\degreeCelsius}. Cultures were grown for 6-9 hours post-induction or overnight and then spun down (optimal yields were observed from shorter durations of growth). All cultures were grown in standard LB in \SI{50}{\micro\gram/\liter} kanamycin and \SI{25}{\micro\gram/\liter} chloramphenicol. Pellets were resuspended in lysis buffer (\SI{20}{\milli\Molar} Tris pH 7.4, \SI{500}{\milli\Molar} NaCl, \SI{10}{\milli\Molar} imidazole, 1 EDTA-free protease inhibitor tablet per \SI{50}{\milli\liter}) and frozen at \SI{-80}{\degreeCelsius}. Frozen suspensions were thawed, sonicated on ice (5 min at 1 s on/1 s off), and clarified (15,000 \textit{g}, 45 min, \SI{4}{\degreeCelsius}). The clarified supernatant was filtered (\SI{0.2}{\micro\meter}), and calsequestrin-containing fractions were isolated by IMAC using a 5 mL HisTrap FF column attached to a GE Akta FPLC (IMAC Buffer A: 20 mM Tris pH 7.4, 500 mM NaCl, 10 mM imidazole; IMAC Buffer B: 20 mM Tris pH 7.4, 500 mM NaCl, 300 mM imidazole). Protein was eluted in 10\% steps of Buffer B. The first eluted fraction (10\% Buffer B) was always discarded (consistently observed to be impure as determined by SDS-PAGE). Remaining protein-containing fractions were pooled. TEV protease was added to the pooled fractions at a concentration of 1:40 by mass, and the protein was dialyzed overnight at \SI{4}{\degreeCelsius} in TEV protease dialysis buffer (50 mM Tris pH 8.0, 0.5 mM EDTA, 1 mM DTT). The cleaved protein was further dialyzed for several hours in EDTA dialysis buffer (20 mM HEPES pH 7.3, 100 mM NaCl, 5 mM EDTA) and then overnight into Anion Exchange Buffer A (20 mM HEPES pH 7.3, 100 mM NaCl). Anion exchange polishing was performed using a HisTrap FF column in series with 3x1 mL Mono Q columns. (Buffer A: 20 mM HEPES pH 7.3, 100 mM NaCl; Buffer B: 20 mM HEPES pH 7.3, 1 M NaCl). Protein was eluted in a continuous gradient up to 100\% Buffer B), with calsequestrin-rich fractions consistently eluting at 40-50\% Buffer B. Fractions were collected and analyzed for purity by SDS-PAGE and A260/A280 ratio. Fractions that were optimally pure and free of A260 contamination were pooled, concentrated to ~20 mg/mL, and frozen at \SI{-80}{\degreeCelsius}.

Alanine mutants were purified as described above, except that phosphate IMAC buffers were employed (Buffer A: 20 mM phosphate buffer at pH 7.4, 500 mM NaCl, 10 mM imidazole; Buffer B: 20 mM phosphate buffer at pH 7.4, 500 mM NaCl, 300 mM imidazole), and an on-column high-salt wash was performed (20 mM phosphate buffer at pH 7.4, 2 M NaCl). In addition, the TEV protease dialysis buffer used for these purifications contained 100 mM NaCl.

\subsection*{Crystallization of Cardiac Calsequestrin}
Crystallization screens were carried out in 96-well hanging-drop format and monitored using a Formulatrix Rock Imager automated imaging system. Conditions conducive to crystal growth were optimized and then reproduced in a 24-well format. The best diffraction was obtained by mixing thawed protein (\SIrange[range-units = single]{10}{20}{\mg\per\milli\liter} in \SI{20}{\milli\Molar} HEPES pH 7.3, \SIrange[range-units = single]{400}{500}{\milli\Molar} NaCl) 1:1 with 15\% PEG 4000 and \SI{400}{\milli\Molar} \ch{Li2SO4}. The pH of the PEG 4000 solution used to produce the best-diffracting crystals was tested by litmus paper and found to be approximately 3-3.5. Despite the presence of 20 mM HEPES in the protein reagent, the pH of the drops in which crystals grew was controlled by the PEG and remained ~3-3.5. Freshly made PEGs were incompatible with calsequestrin crystal growth except when concentrated HCl was added to the mother liquor, producing crystals similar to those observed with benchtop-aged PEGs. Interestingly, only unbuffered conditions yielded crystals. Multiple attempts to grow crystals at a buffered low pH (using acetate or glycine-based buffers) failed.

\subsection*{Ytterbium Soak of Cardiac Calsequestrin Crystals}
We initially attempted to identify calcium sites using anomalous signal from calcium (\ch{CaCl2}) added to the crystallization condition described above. We were unsuccessful, likely due to a combination of several factors. The calcium absorption edge is unreachable with conventional tunable x-ray sources and in normal atmosphere; thus, it can only be approached, with resulting weakened anomalous signal. In addition, calsequestrin has an average \textit{K}\textsubscript{d} for calcium of ~\SI{1}{\milli\Molar}. Thus, occupancy at a typical site would be expected to be lower as compared to other calcium-binding proteins. Presence of sulfate in the crystallization condition limited calcium concentrations to approximately \SI{14}{\milli\Molar} and below, above which a precipitate was observed. Although this limit is above the \textit{K}\textsubscript{d}, it was insufficient for robust anomalous signal. Crystals of calsequestrin that formed in trace calcium were therefore soaked in \ch{YbCl3}. Hanging drops containing calsequestrin crystals were uncovered and an Eppendorf Microloader was used to inject \SI{2}{\micro\liter} drops (\SI{1}{\micro\liter} protein and \SI{1}{\micro\liter} mother liquor) with \SI{200}{\micro\liter} of \SI{2}{\Molar} \ch{YbCl3}. Data were collected within 5 minutes with no back-soaking.

\subsection*{Crystal Data Collection and Structure Determination}
Hanging drops were uncovered and submerged in a drop of Parabar 10312 (previously known as Paratone). For Yb soaks, Yb was quickly injected prior to application of the oil. Crystals were looped and pulled through the oil. Excess oil was blotted away, and the loop was mounted directly into the cryostream of the Tom-Alber-Tron endstation at ALS beamline 8.3.1. Frames were collected at \SI{1.116}{\keV} (\SI{1.386}{\keV} for Yb-soaked crystals) using the endstation's Pilatus3 S 6M detector with a strategy designed to balance redundancy against radiation damage. 

\subsection*{Structure Determination}
Diffraction images were processed with xia2 using the DIALS integration pipeline and a resolution cutoff of CC\textsubscript{1/2} > 0.3 \supercite{Winter2018-oa,Winter2010-tx,Evans2013-wu,Evans2006-cc,Winn2011-fi}. For the native structure, the merged diffraction intensities were used to find a molecular replacement solution in Phaser \supercite{McCoy2007-rr} with the previously published canine cardiac calsequestrin structure, 1SJI \supercite{Park2004-bu}, serving as the initial model. This resulted in a solution in space group P4\textsubscript{3}22 containing one calsequestrin chain per AU. This solution was refined in PHENIX \supercite{Adams2010-qs} with PHENIX AutoBuild \supercite{Adams2010-qs,Afonine2012-xn,Terwilliger2004-hs,Terwilliger2008-zf,Zwart2005-kc}, with extensive manual model-building in Coot \supercite{Emsley2010-il}. For the Yb-complexed dataset, data were processed as above but with preservation of anomalous signal (no merging of Friedel pairs). The refined native structure was used a molecular replacement model, and a solution was found in space group P4\textsubscript{3}2\textsubscript{1}2 containing a dimer in the AU. This solution refined poorly. The Yb-complexed dataset was reprocessed in P1, and a molecular replacement solution was found in P1 with 16 chains in the AU. Refinement of this model was tested using Zanuda \supercite{Lebedev2014-po}, and the best R-free was found to be in space group P12\textsubscript{1}1. The Yb-complexed dataset was reprocessed in space group P12\textsubscript{1}1, and a molecular replacement solution was found with 8 chains in the AU. This solution refined well. The anomalous map was used in refinement to place Yb atoms in the structure.

\subsection*{Turbidity Assays}
Recombinant protein samples were thawed, diluted in \SIrange{2}{3}{\milli\liter} of Turbidity Assay Buffer (\SI{15}{\milli\Molar} Tris pH 7.4, \SI{20}{\milli\Molar} NaCl, \SI{85}{\milli\Molar} KCl) and dialyzed in Turbidity Assay Buffer plus \SI{10}{\milli\Molar} EDTA. Samples were then re-dialyzed overnight in the same buffer without EDTA. Protein A280 was measured in triplicate (Nanodrop) using the appropriate matching buffer as background, and protein was diluted to \SI{2.25}{\micro\Molar} in a \SI{140}{\micro\liter} volume in half-area wells of a \textmu Clear 96-well plate. The plate was covered, and protein in the wells was allowed to equilibrate on the benchtop for 20 minutes. The turbidity assay was performed using a BioTek Synergy 2 plate reader equipped with reagent injectors. Seven \si{\micro\liter} of \SI{20}{\milli\Molar} \ch{CaCl2} solution was injected into each well for a final concentration of approximately \SI{1}{\milli\Molar}, and the plate underwent shaking for \SI{20}{\second}. Absorbance at \SI{350}{\nm} was monitored for \SI{45}{\minute}. The protocol was performed in plate synchronized mode for consistent well-to-well timing. A \SI{100}{\milli\Molar} ion-selective electrode calcium standard (Sigma, cat no. 21059) stock solution was used for all \ch{CaCl2} dilutions.

In the assessment of the response of calsequestrin assemblies to EDTA, slightly different conditions were used. The Turbidity Assay Buffer contained Tris at (\SI{15}{\milli\Molar} Tris pH 7.2 and contained no potassium. This assay was performed in a black plate with \textmu Clear bottom. The well-synchronized plate reader mode was used to rapidly measure the effect of EDTA addition. After addition of \ch{CaCl2} to a final concentration of \SI{1}{\milli\Molar}, absorbance at \SI{350}{\nm} was monitored. Monitoring of absorbance continued as EDTA was then added to a final concentration of \SI{1}{\milli\Molar} using the plate reader's reagent injector module.

\subsection*{Continuum Electrostatics}
Atomic coordinates of a calsequestrin dimer-dimer complex were prepared using PDB2PQR \supercite{Dolinsky2004-oa}. The Assisted Model Building with Energy Refinement (AMBER) force field was used \supercite{Cornell1995-ky}. Electrostatic calculations were performed by The Adaptive Poisson-Boltzmann Solver (APBS) \supercite{Jurrus2018-gc} using the nonlinear Poisson-Boltzmann equation. Relevant conditions were a protein relative dielectric of 2, a solvent dielectric of 78.5, \SI{150}{\milli\Molar} monovalent cation/anion, and a temperature of \SI{25}{\degreeCelsius}.

\subsection*{Visualization and Statistical Analysis}
Protein structure figures were generated using PyMOL \supercite{PyMOL}. The interior cavity of the calsequestrin filament was traced using HOLLOW \supercite{Ho2008-og}. Sequence alignments were generated using \TeXshade \supercite{Beitz2000-lf}. Plots were generated using python matplotlib \supercite{Hunter:2007}. Data points in figures represent mean values, with error bars representing standard deviation. All turbidity assay data points are mean of 3 technical replicates.
% XIA2 0.5.878-g9910dc71-dials-1.14; DIALS 1.14.1-gc065839d3-release; CCP4 7.0.072 (includes Phaser, Coot, Zanuda, etc); PHENIX 1.15.2-3472 (includes Autobuild, phenix.refine, etc); Python 2.7.15 (includes matplotlib); PyMOL 2.2.3; HOLLOW 1.1; TeXshade 1.25

\subsection*{Reporting Summary}
Further information on experimental design is available in the Nature Research Reporting Summary linked to this article.