Calcium homeostasis is essential to cardiac and skeletal muscle physiology, where contractile function requires tight but dynamic control of calcium levels across different cellular compartments. The major calcium storage protein in muscle tissues is calsequestrin, a highly acidic protein responsible for buffering up to \SI{50}{\percent} of total sarcoplasmic reticulum (SR) calcium. Mutations in cardiac calsequestrin cause a highly lethal familial arrhythmia, catecholaminergic polymorphic ventricular tachycardia (CPVT), while mutations in skeletal muscle calsequestrin have been linked to myopathies. Calsequestrin's high density calcium storage is facilitated by homomultimerization of the protein into filaments, but a compelling atomic-resolution structure of a calsequestrin filament is lacking. This gap in knowledge has limited our understanding of calsequestrin biochemistry, SR calcium storage, and molecular mechanisms of calseqestrin-associated diseases. We report here a crystal structure of a cardiac calsequestrin filament with supporting mutation analysis provided by an \textit{in vitro} filamentation assay. We also report and characterize a novel disease-associated mutation, S173I, which localizes to the structure's filament-forming interface. In addition, we show that a previously reported dominant calsequestrin-associated CPVT mutation, K180R, maps to the same multimerization surface. Both mutations disrupt filamentation in vitro, suggesting a model where dominant disease arises from mutations that disrupt multimer formation. Finally, a ytterbium-derivatized structure pinpoints multiple credible calcium sites at filament-forming interfaces, explaining the atomic basis of calsequestrin filamentation in the presence of calcium. This work advances our basic understanding of calsequestrin biochemistry and also provides a unifying structure-function molecular mechanism by which dominant-acting calsequestrin mutations provoke lethal arrhythmias.