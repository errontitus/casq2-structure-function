\begin{figure}
\figuretitle{Figure~\ref{fig:filament_overview}}
\begin{fullpanelvar}
    \begin{emptypanel}{}
        \node(filament)[inner sep=0pt,below right]{\includegraphics[width=\linewidth,height=1.25in,keepaspectratio]{../results/filament_overview/output/overview_surface_cropped.png}};
        %
        \node(captionA)[inner sep=0pt,above left] at (filament.north west) {\normalsize\textbf{\figurepanela}};
        %
        \path (filament.south)++(-1,3.15) coordinate (filament_dimer_NW);
        \path (filament.south)++(1,1) coordinate (filament_dimer_SE);
        %
        \path (filament.south)++(-1,3.15) coordinate (filament_tetramer_NW);
        \path (filament.south)++(2.7,0.5) coordinate (filament_tetramer_SE);
        %
        \node(dimerrectfilament) [fit={(filament_dimer_NW) (filament_dimer_SE)}, dashedrectanglefit] {};
        %
        \node(dimer_dimer_rect_filament) [fit={(filament_tetramer_NW) (filament_tetramer_SE)}, dashedrectanglefit] {};
        %
        \node(filament_caption)[above, inner sep=7pt] at (filament.north) {Cardiac Calsequestrin Filament};
        %
        %
        %
        \node(dimer)[inner sep=0pt,right=1.5cm of filament]{\includegraphics[width=\linewidth,height=1.25in,keepaspectratio]{../results/filament_overview/output/overview_dimer_cropped.png}};
        %
        \node(dimerrect) [fit=(dimer), dashedrectanglefit] {};
        \node(dimer_caption)[above, inner sep=7pt] at (dimer.north) {Dimer};
        %        
        \node(dimer_chainAlabel)[above, inner sep=0pt] at ([shift={(-0.125cm,-0.6cm)}]dimer.south west) {Chain A};
        \draw[] ([shift={(0cm,0.125cm)}]dimer_chainAlabel.north) -- ([shift={(0.25cm,0.125cm)}]dimer.south west);
        %
        \node(dimer_chainBlabel)[above, inner sep=0pt] at ([shift={(0.125cm,-0.6cm)}]dimer.south east) {Chain B};
        \draw[] ([shift={(0cm,0.125cm)}]dimer_chainBlabel.north) -- ([shift={(-0.25cm,0.125cm)}]dimer.south east);
        % \path (dimer.south west)++(0,-0.5) coordinate (chainA);
        % \path (dimer.south east)++(0,-0.5) coordinate (chainB);
        % %
        % \node(chainAlabel) [above, inner sep=2pt, align=center] at (chainA) {Chain A};
        % \node(chainBlabel) [above, inner sep=2pt, align=center] at (chainB) {Chain B};
        % %
        %
        %
        \node(dimer_dimer)[inner sep=0pt,below=4.5cm of filament.west,anchor=west]{\includegraphics[width=\linewidth,height=1.5in,keepaspectratio]{../results/filament_overview/output/overview_tetramer_cropped.png}};
        %
        \node(dimer_dimer_rect) [fit=(dimer_dimer), dashedrectanglefit] {};
        %
        % \path (dimer_dimer.north east)++(0,0) coordinate (dimer_dimer_chainA);
        % \path (dimer_dimer.south east)++(0.75,1) coordinate (dimer_dimer_chainB);
        % \path (dimer_dimer.north)++(1.5,-0.5) coordinate (dimer_dimer_chainAprime);
        % \path (dimer_dimer.south east)++(0.75,1) coordinate (dimer_dimer_chainBprime);
        %
        % \node(dimer_dimer_chainAlabel)[above, inner sep=0pt] at ([shift={(-0.25cm,0.75cm)}]dimer_dimer.north west) {Chain A};
        % \draw[] ([shift={(0cm,-0.25cm)}]dimer_dimer_chainAlabel.south) -- ([shift={(0.25cm,-0.125cm)}]dimer_dimer.north west);
        % %
        % \node(dimer_dimer_chainBlabel)[above, inner sep=0pt] at ([shift={(-0.25cm,0.5cm)}]dimer_dimer.north) {Chain B};
        % \draw[] ([shift={(0cm,-0.25cm)}]dimer_dimer_chainBlabel.south) -- ([shift={(-0.25cm,-0.125cm)}]dimer_dimer.north);
        %
        \node(dimer_dimer_chainAprimelabel)[above, inner sep=2pt] at ([shift={(0.9cm,0.75cm)}]dimer_dimer.east) {Chain A'};
        \draw[] ([shift={(0cm,0cm)}]dimer_dimer_chainAprimelabel.west) -- ([shift={(-0.25cm,0.5cm)}]dimer_dimer.east);
        %
        \node(dimer_dimer_chainBprimelabel)[above, inner sep=2pt] at ([shift={(0.9cm,-0.5cm)}]dimer_dimer.south east) {Chain B'};
        \draw[] ([shift={(0cm,0cm)}]dimer_dimer_chainBprimelabel.west) -- ([shift={(-0.125cm,0.125cm)}]dimer_dimer.south east);
        % \node(dimer_dimer_chainAlabel) [above, inner sep=2pt, align=center] at (dimer_dimer_chainA) {Chain A};
        % \node(dimer_dimer_chainBlabel) [above, inner sep=2pt, align=center] at (dimer_dimer_chainB) {Chain B};
        % \node(dimer_dimer_chainAprimelabel) [above, inner sep=2pt, align=center] at (dimer_dimer_chainAprime) {Chain A'};
        % \node(dimer_dimer_chainBprimelabel) [above, inner sep=2pt, align=center] at (dimer_dimer_chainBprime) {Chain B'};
        %
        %
        %
        \path[-] (dimerrectfilament.east) edge (dimerrect.west);
        \path[-] (dimer_dimer_rect_filament.south) edge (dimer_dimer_rect.north);
        %
        % Beginning of thioredoxin stuff
        %
        \node(filament_thio)[inner sep=0pt,below=5cm of dimer_dimer.west,anchor=west]{\includegraphics[width=\linewidth,height=1.25in,keepaspectratio]{../results/filament_overview/output/overview_surface_thioredoxins_cropped.png}};
        %
        \node(captionA)[inner sep=0pt,above left] at (filament_thio.north west) {\normalsize\textbf{\figurepanelb}};
        %
        \path (filament_thio.south)++(-1,1.1) coordinate (thioprotomerSW);
        \path (filament_thio.south)++(1,3.1) coordinate (thioprotomerNE);
        %
        \node(protomerrectfilament) [fit={(thioprotomerSW) (thioprotomerNE)}, dashedrectanglefit] {};
        %
        \node(filament_caption)[above, inner sep=7pt] at (filament_thio.north) {Cardiac Calsequestrin Filament (Colored by Thioredoxin Domain)};
        %
        %
        %
        \node(monomers)[inner sep=0pt,right=1.5cm of filament_thio]{\includegraphics[width=\linewidth,height=1.25in,keepaspectratio]{../results/filament_overview/output/overview_monomers_cropped.png}};
        %
        \node(protomerrect) [fit={(monomers.south) (monomers.north west)}, dashedrectanglefit] {};
        %
        \node(domainI)[above, inner sep=0pt] at ([shift={(2cm,1cm)}]monomers.north) {Domain I};
        \draw[] ([shift={(0cm,-0.25cm)}]domainI.south) -- ([shift={(2.25cm,0.125cm)}]monomers.north);
        %
        \node(domainII)[below left, inner sep=0pt] at ([shift={(0.8cm,-0.8cm)}]monomers.south) {Domain II};
        \draw[] ([shift={(0cm,0.25cm)}]domainII.north) -- ([shift={(1cm,0.15cm)}]monomers.south);
        %
        \node(domainIII)[below left, inner sep=0pt] at ([shift={(0cm,-0.8cm)}]monomers.south east) {Domain III};
        \draw[] ([shift={(0cm,0.25cm)}]domainIII.north) -- ([shift={(-0.6cm,-0.2cm)}]monomers.south east);
        %
        %
        \path (monomers.north)++(-0.75,-1) coordinate (domains);
        %
        %
        %
        %
        \node(filament_thio_inner)[inner sep=0pt,below=4.5cm of filament_thio.west,anchor=west]{\includegraphics[width=\linewidth,height=1.25in,keepaspectratio]{../results/filament_overview/output/overview_surface_thioredoxins_23_domain_colors_cropped.png}};
        %
        \node(captionB)[inner sep=0pt,above left] at (filament_thio_inner.north west) {\normalsize\textbf{\figurepanelc}};
        %
        \node(filament_caption)[above, inner sep=7pt] at (filament_thio_inner.north) {Thioredoxin Domain II/III Double Helix};
        %
        %
        %
        \path[-] (protomerrect.north west) edge (protomerrectfilament.east);
        %    
    \end{emptypanel}
\end{fullpanelvar}
\caption[Overview of the cardiac calsequestrin filament]{\textbf{\headingsubsectiontwo}. \figurepanelcaptiona The cardiac calsequestrin candidate filament (PDB ID 6OWV) is shown, along with a representative dimeric and tetrameric assembly. Dimers are stacked on a screw axis with 90 degrees of rotation per dimer. \figurepanelcaptionb The helical character of the filament is revealed at the domain level. Cardiac calsequestrin monomers are colored by thioredoxin domain (domain I, purple; domain I, cyan; domain III, yellow). Viewed at the level of its thioredoxin domains (3 per protomer), the filament consists of an inner thioredoxin double helix (domains II and III) with an outer thioredoxin single helix (domain I) wrapping the double helical core. Right side: The monomers are translated but remain in their dimer-forming orientation. \figurepanelcaptionc The inner double helix of the filament consisting of thioredoxin domains II and III.}
\label{fig:filament_overview}
\end{figure}